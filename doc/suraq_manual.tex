\documentclass[a4paper]{article}

\usepackage{url}
\usepackage{xspace}
\usepackage{booktabs}
\usepackage{graphicx}

%%%%%%%%%%%%%%%%%%%%%%%%%%%%%%%%%%%%%%%%%%%%%%%%%%%%%%%%%
% Algorithm2e package and some custom definitions for it
\usepackage[boxruled, linesnumbered]{algorithm2e}
\SetAlgorithmName{Listing}{Listing}{Listings}
\SetKw{declareFun}{declare-fun}
\SetKw{defineFun}{define-fun}
\SetKw{setLogic}{set-logic}
\SetKwData{suraqlogic}{Suraq}
\SetKw{Array}{Array}
\SetKw{Bool}{Bool}
\SetKw{Control}{Control}
\SetKw{Value}{Value}
\SetKwData{true}{true}
\SetKwData{false}{false}
\SetKw{store}{store}
\SetKw{select}{select}
\SetKw{assert}{assert}
\SetKw{smtforall}{forall}

%%%%%%%%%%%%%%%%%%%%%%%%%%%%%%%%%%%%%%%%%%%%%%%%%%%%%%%%%

\newcommand\suraq{\mbox{\textsc{Suraq}}\xspace}
\newcommand\textbful[1]{\textbf{\underline{#1}}}

\bibliographystyle{plain}
\begin{document}

\title{Suraq User's Manual\thanks{This work was supported in part by the
European Commission through project DIAMOND (FP7-2009-IST-4-248613),
and the Austrian Science Fund (FWF) through the national research
network RiSE (S11406-N23).}}

\author{Georg Hofferek\\
Institute for Applied Information Processing\\and Communications
(IAIK),\\
Graz University of Technology, Austria}
\date{}
\maketitle
 \vspace{14cm}
 \pagebreak
% \tableofcontents
% \listofalgorithms


\section{Introduction} \label{sec:introduction}

Welcome to \suraq, the \emph{\textbful{S}ynthesizer using
\textbful{U}ninterpreted Functions, A\textbful{r}rays \textbful{a}nd
E\textbful{q}uality}.\footnote{Any similarity to the name of the
ancient Vulcan logician ``Surak'' \cite{Memory_Alpha_Surak} is
completely coincidental. ;-)} \suraq is a tool for
correct-by-construction controller synthesis. It takes a (logical)
correctness criterion as input, where control signals are
existentially quantified. It then constructs functions for these
control signals so that the specification is fulfilled. The
correctness criterion is expressed as a formula in the logic of
arrays, uninterpreted functions, and equality, with limited
quantification. The theory behind the synthesis approach is presented
in \cite{Hoffer11}.


\section{Input Format} \label{sec:input_format}

The input format used by \suraq is based on SMTLIB version 2.0
\cite{SMTLIB}. However, \suraq imposes some restrictions on the
input, and makes some implicit assumptions; both of which will be
detailed in section. A \suraq input file forms the specification for
a controller synthesis problem by stating a so-called
\emph{equivalence criterion} \cite{Hoffer11}, which is a formula in a
logic with (extensional) arrays, uninterpreted functions, and
equality. A (partial) example of what the input looks like is given
in Listing~\ref{list:first_example}. For details on the syntax of
s-expressions, comments, etc., refer to the SMTLIB standard
\cite{SMTLIB}.

\begin{algorithm}
\DontPrintSemicolon
  (\setLogic \suraqlogic)\;
  \BlankLine
  [...]
  \BlankLine
  (\declareFun REGci\_\_ () (\Array \Value \Value))\;
  (\declareFun REGsc\_  ()  (\Array \Value \Value))\;
  (\declareFun REGsc\_\_ () (\Array \Value \Value))\;
\BlankLine
  (\declareFun v    () \Value)\;
  (\declareFun v\_  () \Value)\;
  (\declareFun w    () \Value)\;
  (\declareFun s    () \Value)\;
\BlankLine
  (\declareFun x () \Control)\;
\BlankLine
  (\declareFun ALU (\Value) \Value)\;
\BlankLine
  (\defineFun equiv \;\Indp
  ((A (\Array \Value \Value))(B (\Array \Value \Value)))\;
  \Bool (\;\Indp
    \smtforall ((i \Value)) (\;\Indp
      = (\select A i)\; \Indp
        (\select B i)\;\Indm
 \Indm      )\;
 \Indm    )\;
 \Indm  )\;
  \BlankLine
  [...]
  \BlankLine
  (\assert \;\Indp
  (formula REG REGci\_  REGci\_\_  REGsc\_  REGsc\_\_  v v\_  w s dest
  x))\;\Indm
  \caption{Example of input format.} \label{list:first_example}
\end{algorithm}


The first command in each \suraq input file needs to be (\setLogic
\suraqlogic). This command identifies the rest of the file as a
\suraq input file. This implicitly defines two sorts\footnote{
``sort'' is the SMTLIB word for a datatype.}, \Control and \Value, in
addition to the sort \Bool, defined by the core theory. \Control is a
sort that is used for Boolean variables, for which logic should be
synthesized. An expression of sort \Control can be used anywhere a
\Bool expression is expected. \Value, on the other hand, is a sort
that represents elements of an uninterpreted domain. It is implicitly
assumed that the domain of sort \Value is large enough to assign
pairwise distinct values to all variables in the entire formula. It
is easiest to think about it as having an infinite domain.
%
It is possible to build arrays of sort \Value, i.e., the (complex)
sort (\Array \Value \Value). Note that sort \Value is used as both,
the index and the element domain of these arrays. Any other sorts are
not supported by \suraq at this time.

After the initial \setLogic command, you should declare all
identifiers that will be used in the formula. Note that SMTLIB
requires that identifiers are declared before their respective first
use. Identifiers include domain variables (sort \Value), Boolean
control variables for which logic should be synthesized (sort
\Control), other Boolean variables (sort \Bool), uninterpreted
functions (mapping from (\Value, \Value, \ldots) to \Value), and
arrays.

\bibliography{suraq}
\end{document}
