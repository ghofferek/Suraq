\documentclass[a4paper]{article}

\usepackage{url}
\usepackage{xspace}
\usepackage{booktabs}
\usepackage{graphicx}


\newcommand\suraq{\mbox{\textsc{Suraq}}\xspace}
\newcommand\textbful[1]{\textbf{\underline{#1}}}

\bibliographystyle{plain}
\begin{document}

\title{Suraq User's Manual\thanks{This work was supported in part by the
European Commission through project DIAMOND (FP7-2009-IST-4-248613),
and the Austrian Science Fund (FWF) through the national research
network RiSE (S11406-N23).}}

\author{Georg Hofferek\\
Institute for Applied Information Processing\\and Communications
(IAIK),\\
Graz University of Technology, Austria}
\date{}
\maketitle


\section{Introduction} \label{sec:introduction}

Welcome to \suraq, the \emph{\textbful{S}ynthesizer using
\textbful{U}ninterpreted Functions, A\textbful{r}rays \textbful{a}nd
E\textbful{q}uality}.\footnote{Any similarity to the name of the
ancient Vulcan logician ``Surak'' \cite{Memory_Alpha_Surak} is
completely coincidental. ;-)} \suraq is a tool for
correct-by-construction controller synthesis. It takes a (logical)
correctness criterion as input, where control signals are
existentially quantified. It then constructs functions for these
control signals so that the specification is fulfilled. The
correctness criterion is expressed as a formula in the logic of
arrays, uninterpreted functions, and equality, with limited
quantification. The theory behind the synthesis approach is presented
in \cite{Hoffer11}.


\bibliography{suraq}
\end{document}
